% Options for packages loaded elsewhere
\PassOptionsToPackage{unicode}{hyperref}
\PassOptionsToPackage{hyphens}{url}
\PassOptionsToPackage{dvipsnames,svgnames,x11names}{xcolor}
%
\documentclass[
]{article}

\usepackage{amsmath,amssymb}
\usepackage{iftex}
\ifPDFTeX
  \usepackage[T1]{fontenc}
  \usepackage[utf8]{inputenc}
  \usepackage{textcomp} % provide euro and other symbols
\else % if luatex or xetex
  \usepackage{unicode-math}
  \defaultfontfeatures{Scale=MatchLowercase}
  \defaultfontfeatures[\rmfamily]{Ligatures=TeX,Scale=1}
\fi
\usepackage{lmodern}
\ifPDFTeX\else  
    % xetex/luatex font selection
\fi
% Use upquote if available, for straight quotes in verbatim environments
\IfFileExists{upquote.sty}{\usepackage{upquote}}{}
\IfFileExists{microtype.sty}{% use microtype if available
  \usepackage[]{microtype}
  \UseMicrotypeSet[protrusion]{basicmath} % disable protrusion for tt fonts
}{}
\makeatletter
\@ifundefined{KOMAClassName}{% if non-KOMA class
  \IfFileExists{parskip.sty}{%
    \usepackage{parskip}
  }{% else
    \setlength{\parindent}{0pt}
    \setlength{\parskip}{6pt plus 2pt minus 1pt}}
}{% if KOMA class
  \KOMAoptions{parskip=half}}
\makeatother
\usepackage{xcolor}
\usepackage[lmargin=1in,tmargin=0.5in]{geometry}
\setlength{\emergencystretch}{3em} % prevent overfull lines
\setcounter{secnumdepth}{-\maxdimen} % remove section numbering
% Make \paragraph and \subparagraph free-standing
\makeatletter
\ifx\paragraph\undefined\else
  \let\oldparagraph\paragraph
  \renewcommand{\paragraph}{
    \@ifstar
      \xxxParagraphStar
      \xxxParagraphNoStar
  }
  \newcommand{\xxxParagraphStar}[1]{\oldparagraph*{#1}\mbox{}}
  \newcommand{\xxxParagraphNoStar}[1]{\oldparagraph{#1}\mbox{}}
\fi
\ifx\subparagraph\undefined\else
  \let\oldsubparagraph\subparagraph
  \renewcommand{\subparagraph}{
    \@ifstar
      \xxxSubParagraphStar
      \xxxSubParagraphNoStar
  }
  \newcommand{\xxxSubParagraphStar}[1]{\oldsubparagraph*{#1}\mbox{}}
  \newcommand{\xxxSubParagraphNoStar}[1]{\oldsubparagraph{#1}\mbox{}}
\fi
\makeatother


\providecommand{\tightlist}{%
  \setlength{\itemsep}{0pt}\setlength{\parskip}{0pt}}\usepackage{longtable,booktabs,array}
\usepackage{calc} % for calculating minipage widths
% Correct order of tables after \paragraph or \subparagraph
\usepackage{etoolbox}
\makeatletter
\patchcmd\longtable{\par}{\if@noskipsec\mbox{}\fi\par}{}{}
\makeatother
% Allow footnotes in longtable head/foot
\IfFileExists{footnotehyper.sty}{\usepackage{footnotehyper}}{\usepackage{footnote}}
\makesavenoteenv{longtable}
\usepackage{graphicx}
\makeatletter
\newsavebox\pandoc@box
\newcommand*\pandocbounded[1]{% scales image to fit in text height/width
  \sbox\pandoc@box{#1}%
  \Gscale@div\@tempa{\textheight}{\dimexpr\ht\pandoc@box+\dp\pandoc@box\relax}%
  \Gscale@div\@tempb{\linewidth}{\wd\pandoc@box}%
  \ifdim\@tempb\p@<\@tempa\p@\let\@tempa\@tempb\fi% select the smaller of both
  \ifdim\@tempa\p@<\p@\scalebox{\@tempa}{\usebox\pandoc@box}%
  \else\usebox{\pandoc@box}%
  \fi%
}
% Set default figure placement to htbp
\def\fps@figure{htbp}
\makeatother

\usepackage{titling}
\pretitle{\begin{flushleft}\LARGE}
\posttitle{\end{flushleft}}
\preauthor{\begin{flushleft}\small}
\postauthor{\end{flushleft}\vspace{-2em}}
\renewcommand{\maketitlehooka}{\vspace{-2em}}
\makeatletter
\@ifpackageloaded{caption}{}{\usepackage{caption}}
\AtBeginDocument{%
\ifdefined\contentsname
  \renewcommand*\contentsname{Table of contents}
\else
  \newcommand\contentsname{Table of contents}
\fi
\ifdefined\listfigurename
  \renewcommand*\listfigurename{List of Figures}
\else
  \newcommand\listfigurename{List of Figures}
\fi
\ifdefined\listtablename
  \renewcommand*\listtablename{List of Tables}
\else
  \newcommand\listtablename{List of Tables}
\fi
\ifdefined\figurename
  \renewcommand*\figurename{Figure}
\else
  \newcommand\figurename{Figure}
\fi
\ifdefined\tablename
  \renewcommand*\tablename{Table}
\else
  \newcommand\tablename{Table}
\fi
}
\@ifpackageloaded{float}{}{\usepackage{float}}
\floatstyle{ruled}
\@ifundefined{c@chapter}{\newfloat{codelisting}{h}{lop}}{\newfloat{codelisting}{h}{lop}[chapter]}
\floatname{codelisting}{Listing}
\newcommand*\listoflistings{\listof{codelisting}{List of Listings}}
\makeatother
\makeatletter
\makeatother
\makeatletter
\@ifpackageloaded{caption}{}{\usepackage{caption}}
\@ifpackageloaded{subcaption}{}{\usepackage{subcaption}}
\makeatother

\usepackage{bookmark}

\IfFileExists{xurl.sty}{\usepackage{xurl}}{} % add URL line breaks if available
\urlstyle{same} % disable monospaced font for URLs
\hypersetup{
  pdftitle={PRACTICE EXAM 2 PART 1},
  colorlinks=true,
  linkcolor={blue},
  filecolor={Maroon},
  citecolor={Blue},
  urlcolor={Blue},
  pdfcreator={LaTeX via pandoc}}


\title{PRACTICE EXAM 2 PART 1}
\usepackage{etoolbox}
\makeatletter
\providecommand{\subtitle}[1]{% add subtitle to \maketitle
  \apptocmd{\@title}{\par {\large #1 \par}}{}{}
}
\makeatother
\subtitle{MATH411}
\author{}
\date{}

\begin{document}
\maketitle


\begin{enumerate}
\def\labelenumi{\arabic{enumi}.}
\item
  \textbf{Direct Solving via Gaussian Elimination / PA=LU Factorization}

  \textbf{(a) Gaussian Elimination}

  Solve the following system of equations using Gaussian elimination:

  \[
  \begin{cases}
  2x + y - z = 8 \\
  -3x - y + 2z = -11 \\
  -2x + y + 2z = -3 \\
  \end{cases}
  \]

  \vspace{5cm}

  \textbf{(b) PA=LU Factorization}

  Perform a PA=LU factorization of the following matrix \(A\):

  \[
  A = \begin{pmatrix}
  0 & 2 & 1 \\
  1 & -2 & -1 \\
  -1 & 0 & 2 \\
  \end{pmatrix}
  \]

  Find the permutation matrix \(P\), lower triangular matrix \(L\), and
  upper triangular matrix \(U\) such that \(PA = LU\).
\end{enumerate}

\newpage

\begin{enumerate}
\def\labelenumi{\arabic{enumi}.}
\setcounter{enumi}{1}
\item
  \textbf{Iterative Methods}

  \textbf{(a) Convergence of Iterative Methods}

  Consider the matrix \(A\):

  \[
  A = \begin{pmatrix}
  5 & -2 & 3 \\
  -3 & 9 & 1 \\
  2 & -1 & -7 \\
  \end{pmatrix}
  \]

  Is the Jacobi iterative method guaranteed to converge for the system
  \(A\mathbf{x} = \mathbf{b}\) for any \(\mathbf{b}\)? Justify your
  answer.

  \vspace{6cm}

  \textbf{(b) Theorem Application}

  Explain how the spectral radius of the iteration matrix affects the
  convergence of an iterative method. Refer to the theorem stating that
  if the spectral radius is less than 1, the method converges.
\end{enumerate}

\newpage

\begin{enumerate}
\def\labelenumi{\arabic{enumi}.}
\setcounter{enumi}{2}
\item
  \textbf{Jacobi and Gauss-Seidel Methods}

  \textbf{(a) Jacobi Method}

  For the following system:

  \[
  \begin{cases}
  10x_1 - x_2 + 2x_3 = 6 \\
  -x_1 + 11x_2 - x_3 + 3x_4 = 25 \\
  2x_1 - x_2 + 10x_3 - x_4 = -11 \\
  3x_2 - x_3 + 8x_4 = 15 \\
  \end{cases}
  \]

  Perform two iterations of the Jacobi method starting with
  \(\mathbf{x}^{(0)} = \begin{pmatrix} 0 \\ 0 \\ 0 \\ 0 \end{pmatrix}\).

  \vspace{7cm}

  \textbf{(b) Gauss-Seidel Method}

  Perform two iterations of the Gauss-Seidel method for the same system
  and initial guess.
\end{enumerate}

\newpage

\begin{enumerate}
\def\labelenumi{\arabic{enumi}.}
\setcounter{enumi}{3}
\item
  \textbf{Least Squares and Orthogonalizations}

  \textbf{(a) Normal Equations}

  Given the overdetermined system:

  \[
  \begin{cases}
  x + y = 2 \\
  2x + y = 3 \\
  x + 2y = 3 \\
  \end{cases}
  \]

  Find the least squares solution by setting up and solving the normal
  equations.

  \vspace{7cm}

  \textbf{(b) QR Factorization}

  Using the same system, perform the QR factorization of the matrix
  \(A\) (the coefficient matrix) using the Gram-Schmidt process (either
  classical or modified). Then, solve \(R\mathbf{x} = Q^\top\mathbf{b}\)
  by back substitution.
\end{enumerate}

\newpage

\begin{enumerate}
\def\labelenumi{\arabic{enumi}.}
\setcounter{enumi}{4}
\item
  \textbf{Gram-Schmidt Process}

  Apply the Gram-Schmidt process to the following set of vectors to
  obtain an orthonormal basis for \(\mathbb{R}^3\):

  \[
  \mathbf{v}_1 = \begin{pmatrix} 1 \\ 1 \\ 0 \end{pmatrix}, \quad
  \mathbf{v}_2 = \begin{pmatrix} 1 \\ 0 \\ 1 \end{pmatrix}, \quad
  \mathbf{v}_3 = \begin{pmatrix} 0 \\ 1 \\ 1 \end{pmatrix}
  \]

  Compute the QR factorization \(A = QR\) where \(A\) has columns
  \(\mathbf{v}_1, \mathbf{v}_2, \mathbf{v}_3\).
\end{enumerate}

\newpage

\begin{enumerate}
\def\labelenumi{\arabic{enumi}.}
\setcounter{enumi}{5}
\item
  \textbf{Solving \(R\mathbf{x} = Q^\top\mathbf{b}\) by Back
  Substitution}

  Given the QR factorization from Question 5, and a vector
  \(\mathbf{b} = \begin{pmatrix} 2 \\ 3 \\ 4 \end{pmatrix}\), solve
  \(R\mathbf{x} = Q^\top\mathbf{b}\) for \(\mathbf{x}\) using back
  substitution.
\end{enumerate}

\newpage

\begin{enumerate}
\def\labelenumi{\arabic{enumi}.}
\setcounter{enumi}{6}
\item
  \textbf{GMRES with Preconditioning}

  \textbf{(a) Conceptual Understanding}

  Explain the role of preconditioning in the GMRES (Generalized Minimal
  Residual) method and how it improves convergence when solving
  non-symmetric linear systems.

  \vspace{7cm}

  \textbf{(b) Computational Exercise}

  Given the matrix \(A\) and vector \(\mathbf{b}\):

  \[
  A = \begin{pmatrix}
  4 & 1 \\
  2 & 3 \\
  \end{pmatrix}, \quad
  \mathbf{b} = \begin{pmatrix}
  1 \\
  0 \\
  \end{pmatrix}
  \]

  Suppose we use a preconditioner \(M\) such that \(M^{-1}A\) has better
  spectral properties. Let

  \[
  M = \begin{pmatrix}
  4 & 0 \\
  0 & 3 \\
  \end{pmatrix}.
  \]

  Perform one iteration of the preconditioned GMRES algorithm starting
  from \(\mathbf{x}_0 = \begin{pmatrix} 0 \\ 0 \end{pmatrix}\).
\end{enumerate}




\end{document}
